\documentclass[a4paper,10pt]{article}
\usepackage[utf8x]{inputenc}
\usepackage{mathpazo,helvet}
\usepackage{amsfonts}
\usepackage{amsmath}
\usepackage{amssymb}
\usepackage{amsthm}

%opening
\title{CATMAID Specifications}
\author{Tobias Pietzsch \and Stephan Saalfeld}


\newcommand{\vc}[1]{\ensuremath{\mathbf{#1}}}
\newcommand{\mat}[1]{\ensuremath{\mathsf{#1}}}

\begin{document}

\maketitle


\section{Introduction}

Recently, CATMAID has been extended to support $n$-dimensional datasets with each dataset being registered into a project space by an affine transformation.  This document aims at clarifying how coordinates are related, which coordinate spaces are used by which control and display, how this change affects the naming scheme for image tiles and `interface' methods of displays and controls.


\section{Coordinate Spaces}

CATMAID uses four distinct coordinate spaces:

\newcommand{\xprj}{\vc{x}_p}
\newcommand{\xstack}{\vc{x}_s}
\newcommand{\xtile}{\vc{x}_t}
\newcommand{\rstack}{\vc{r}}
\newcommand{\Sstack}{\operatorname{diag}(\rstack)}
\newcommand{\Astack}{\mat{A}}
\newcommand{\tstack}{\vc{t}}
\newcommand{\toTile}{\operatorname{tile}}

\begin{description}
\item[project space]
  Each \emph{project} specifies a continuous $n$-dimensional Euclidean space with each of its axes being labeled as a particular physical dimension.
  For the sake of flexibility, we have not specified units to particular physical dimensions but leave that open to be specified by each project individually.
  \begin{equation}
    \xprj = \begin{pmatrix}
        x_1 \\ \vdots \\ x_n
      \end{pmatrix}
  \end{equation}
\item[stack space]
  Each stack specifies a discrete $n$-dimensional pixel space that, usually, is dictated by the resolution of the pixel data in this data set.
  Accordingly, it is related to project space through a resolution vector $\rstack$ that specifies a linear scale of a pixel in stack space into the corresponding physical unit in project space.
  Furthermore, stack space coordinates can be transferred into project space by an optional affine transformation $\Astack, \tstack$.
  Project space coordinates are transferred into stack coordinates by first applying the affine transformation and then scaling as indicated by the resolution vector.
  \begin{equation}
    \xstack = \Sstack \left( \Astack \xprj + \tstack \right)
  \end{equation}
\item[tile space]
  Image data is displayed split into equally sized tiles at a particular scale.
  Tile coordinates thus have an additional discrete scale-level dimension $s$ with $scale = 1/2^s$.
  The first two dimensions, usually $x$ and $y$ in the spatial domain represent the column and row in the grid of tiles on scale-level $s$.
  Let $\xstack = \left( x_1, \dots,  x_n \right)^T$ be stack coordinates of a point and $s$ a desired scale-level.
  Then the corresponding tile coordinates are
  \begin{equation}
    \xtile = \toTile (\xstack) \begin{pmatrix}
      \lfloor 2^{-s} \cdot w_{tile}^{-1} \cdot x_1 \rfloor \\[1mm]
      \lfloor 2^{-s} \cdot h_{tile}^{-1} \cdot x_2 \rfloor \\[1mm]
      x_3 \\
      \vdots \\
      x_n \\
      s
    \end{pmatrix}.
  \end{equation}
  Note that the mapping from stack to tile coordinates depends on the scale-level $s$:
  for every stack coordinate there is a set of corresponding tile coordinates, one for each $s$.
  Given tile coordinates $\xtile = \left( x_1, \dots,  x_n, s \right)^T$ the stack coordinate corresponding
  to the upper-left corner of the tile is
  \begin{equation}
    \xstack = \begin{pmatrix}
      2^{s} \cdot w_{tile} \cdot x_1 \\[1mm]
      2^{s} \cdot h_{tile} \cdot x_2 \\[1mm]
      x_3 \\
      \vdots \\
      x_n
    \end{pmatrix}
  \end{equation}
\item[screen space]
  \begin{equation}
    \vc{x}_r
  \end{equation}
\end{description}

\begin{description}
\item[stack resolution]
  A $n$-dimensional resolution vector is associated with every \emph{stack}, specifying the scaling from stack space to project space along each (stack space) axis.
  \begin{equation}
    \rstack = \begin{pmatrix}
        r_1 \\ \vdots \\ r_n
      \end{pmatrix}
  \end{equation}
  The resolution vector defines a linear scaling matrix $\Sstack$ that relates project and stack coordinates.
\end{description}


\section{Tile Naming Scheme}

Image tiles are addressed by their scale level $s$ with $scale = 1/2^s$ and coordinates in tile space.
They are sorted into directories named as the tile coordinates in reverse order, and with the scale-level inserted between the $z$ and $y$ coordinate.
A tile with coordinates $\xtile = \left( x = 23, y = 42, z = 7, t = 123, s = 0 \right)^T$ would be stored as
\texttt{123/7/0/42/23.jpg}.

\end{document}
